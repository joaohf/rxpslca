% Copyright (c) 2008, João Henrique Ferreira de Freitas
% All rights reserved.
% 
% Redistribution and use in source and binary forms, with or without modification,
% are permitted provided that the following conditions are met:
% 
%     * Redistributions of source code must retain the above copyright notice,
%       this list of conditions and the following disclaimer.
%     * Redistributions in binary form must reproduce the above copyright notice,
%       this list of conditions and the following disclaimer in the documentation and/or 
%       other materials provided with the distribution.
%     * Neither the name of the <ORGANIZATION> nor the names of its contributors may
%       be used to endorse or promote products derived from this software without 
%       specific prior written permission.
% 
% THIS SOFTWARE IS PROVIDED BY THE COPYRIGHT HOLDERS AND CONTRIBUTORS "AS IS" AND ANY 
% EXPRESS OR IMPLIED WARRANTIES, INCLUDING, BUT NOT LIMITED TO, THE IMPLIED WARRANTIES
% OF MERCHANTABILITY AND FITNESS FOR A PARTICULAR PURPOSE ARE DISCLAIMED. IN NO EVENT
% SHALL THE COPYRIGHT OWNER OR CONTRIBUTORS BE LIABLE FOR ANY DIRECT, INDIRECT, INCIDENTAL,
% SPECIAL, EXEMPLARY, OR CONSEQUENTIAL DAMAGES (INCLUDING, BUT NOT LIMITED TO, PROCUREMENT
% OF SUBSTITUTE GOODS OR SERVICES; LOSS OF USE, DATA, OR PROFITS; OR BUSINESS INTERRUPTION)
% HOWEVER CAUSED AND ON ANY THEORY OF LIABILITY, WHETHER IN CONTRACT, STRICT LIABILITY,
% OR TORT (INCLUDING NEGLIGENCE OR OTHERWISE) ARISING IN ANY WAY OUT OF THE USE OF THIS
% SOFTWARE, EVEN IF ADVISED OF THE POSSIBILITY OF SUCH DAMAGE.
% 
% $Id$

\section{Referencial Teórico} \label{subsec:referencial}

Há vários estudos relacionados à descoberta de processos de desenvolvimento dentro de comunidades de SL/CA. Muitos investigam as possibilidades de desvendá-los e ao mesmo tempo melhorá-los, reintroduzindo-os no projeto em questão as melhorias evidenciadas como discutido  em \cite{experience}. 

Entretanto outros estudos procuram mapear todo o processo desde a concepção inicial do projeto até a fase de manutenção a fim de fazer comparativos com processos tradicionais da engenharia de software \cite{multimodal} e também estudá-los. 
Em contrapartida, desenvolvedores de SL/CA também são usuários ou administradores dos software que desenvolvem não havendo uma distinção clara entre usuários e desenvolvedores, como observado tradicionalmente no modelo proprietário de desenvolvimento. Visando justamente esta diferença é que se torna importante levantar meios de usuários também atuarem como desenvolvedores e é este o foco do trabalho.

Qualquer desenvolvedor que busque entrar no projeto raramente encontrará informações específicas em qual parte do processo deve atuar. Utilizando uma exploração sistemática, como demonstrado no trabalho de \cite{issue}, na qual busca analisar as informações presentes nos portais oficiais dos projetos através de uma taxonomia \cite{refframework} dos artefatos (discutido com mais profundidate na seção \ref{sec:materiais}), o desenvolvedor aspirante encontrará uma forma de trabalho aparentemente organizada. 

Todavia, muitos autores defendem a idéia que cada projeto de SL/CA não é igual ao outro, possui sua própria engenharia, processos, métodos e formas de comunicação. Felizmente há pontos semelhantes nos quais podem ser utilizados para traçar guias gerais de contribuição.

% TODO: levantar texto que fala sobre "nenhum projeto é igual ao outro"
% \cite{preliminary}
% \cite{acrossweb}
% \cite{multimodal}
