% Copyright (c) 2008, João Henrique Ferreira de Freitas
% All rights reserved.
% 
% Redistribution and use in source and binary forms, with or without modification,
% are permitted provided that the following conditions are met:
% 
%     * Redistributions of source code must retain the above copyright notice,
%       this list of conditions and the following disclaimer.
%     * Redistributions in binary form must reproduce the above copyright notice,
%       this list of conditions and the following disclaimer in the documentation and/or 
%       other materials provided with the distribution.
%     * Neither the name of the <ORGANIZATION> nor the names of its contributors may
%       be used to endorse or promote products derived from this software without 
%       specific prior written permission.
% 
% THIS SOFTWARE IS PROVIDED BY THE COPYRIGHT HOLDERS AND CONTRIBUTORS "AS IS" AND ANY 
% EXPRESS OR IMPLIED WARRANTIES, INCLUDING, BUT NOT LIMITED TO, THE IMPLIED WARRANTIES
% OF MERCHANTABILITY AND FITNESS FOR A PARTICULAR PURPOSE ARE DISCLAIMED. IN NO EVENT
% SHALL THE COPYRIGHT OWNER OR CONTRIBUTORS BE LIABLE FOR ANY DIRECT, INDIRECT, INCIDENTAL,
% SPECIAL, EXEMPLARY, OR CONSEQUENTIAL DAMAGES (INCLUDING, BUT NOT LIMITED TO, PROCUREMENT
% OF SUBSTITUTE GOODS OR SERVICES; LOSS OF USE, DATA, OR PROFITS; OR BUSINESS INTERRUPTION)
% HOWEVER CAUSED AND ON ANY THEORY OF LIABILITY, WHETHER IN CONTRACT, STRICT LIABILITY,
% OR TORT (INCLUDING NEGLIGENCE OR OTHERWISE) ARISING IN ANY WAY OUT OF THE USE OF THIS
% SOFTWARE, EVEN IF ADVISED OF THE POSSIBILITY OF SUCH DAMAGE.
% 
% $Id$

\section{Introdução} \label{sec:introducao}

O presente trabalho relata um processo e prática para a contribuição e melhoramento em projetos de Software Livre e Código Aberto (SL/CA). Nos posicionamos como desenvolvedor contribuidor, ou seja, aquele que desenvolve esporadicamente e com grandes chances de se tornar um desenvolvedor oficial. Assim pudemos capturar todo o processo de contribuição, interação e desenvolvimento colaborativo, imerso em uma comunidade de Software Livre e Código Aberto.
 
Inicialmente foram levantados três projetos nos quais haviam possibilidades de melhoramentos e criação de novas funcionalidades. Dos três apenas um foi escolhido para ser utilizado como experimentação no qual contribuimos como desenvolvedor de uma extensão para as funcionalidades relacionados a interfaceamento com um SGBD (Sistema de Gerenciamento de Banco de Dados) para o projeto em questão. Assim pudemos anotar as experiências e compilar os resultados contemplados no presente texto.

Através de uma exploração organizada e estruturada foi evidenciado como o desenvolvedor contribuidor pode realizar um papel fundamental para adição de novas funcionalidades ou manutenção do projeto de SL/CA.

O trabalho está organizado na seguinte forma: Na seção \ref{sec:objetivos} listamos os objetivos do trabalho bem com o enfoque dado a pesquisa e exposição prática. A seguir na seção \ref{subsec:referencial} relacionamos e exploramos os trabalhos similares bem como as fontes de pesquisa bibliográfica consultadas. Em \ref{sec:materiais} explanamos como foram feito os experimentos, enquanto que na seção \ref{sec:resultados} discutimos os resultados obtidos e melhores práticas. Em \ref{sec:conclusao} concluímos o trabalho bem como expomos os trabalhos futuros relacionados.

