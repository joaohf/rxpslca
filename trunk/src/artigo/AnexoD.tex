% Copyright (c) 2008, João Henrique Ferreira de Freitas
% All rights reserved.
% 
% Redistribution and use in source and binary forms, with or without modification,
% are permitted provided that the following conditions are met:
% 
%     * Redistributions of source code must retain the above copyright notice,
%       this list of conditions and the following disclaimer.
%     * Redistributions in binary form must reproduce the above copyright notice,
%       this list of conditions and the following disclaimer in the documentation and/or 
%       other materials provided with the distribution.
%     * Neither the name of the <ORGANIZATION> nor the names of its contributors may
%       be used to endorse or promote products derived from this software without 
%       specific prior written permission.
% 
% THIS SOFTWARE IS PROVIDED BY THE COPYRIGHT HOLDERS AND CONTRIBUTORS "AS IS" AND ANY 
% EXPRESS OR IMPLIED WARRANTIES, INCLUDING, BUT NOT LIMITED TO, THE IMPLIED WARRANTIES
% OF MERCHANTABILITY AND FITNESS FOR A PARTICULAR PURPOSE ARE DISCLAIMED. IN NO EVENT
% SHALL THE COPYRIGHT OWNER OR CONTRIBUTORS BE LIABLE FOR ANY DIRECT, INDIRECT, INCIDENTAL,
% SPECIAL, EXEMPLARY, OR CONSEQUENTIAL DAMAGES (INCLUDING, BUT NOT LIMITED TO, PROCUREMENT
% OF SUBSTITUTE GOODS OR SERVICES; LOSS OF USE, DATA, OR PROFITS; OR BUSINESS INTERRUPTION)
% HOWEVER CAUSED AND ON ANY THEORY OF LIABILITY, WHETHER IN CONTRACT, STRICT LIABILITY,
% OR TORT (INCLUDING NEGLIGENCE OR OTHERWISE) ARISING IN ANY WAY OUT OF THE USE OF THIS
% SOFTWARE, EVEN IF ADVISED OF THE POSSIBILITY OF SUCH DAMAGE.
% 
% $Id$

\section{Anexo D: Emails trocados na lista de desenvolvimento do Bacula} \label{sec:anexod}
Neste anexo, apresentamos os principais emails e discussões ocorridas na lista de desenvolvedores do projeto Bacula. 

\subsection{Feedback de desenvolvedores}\label{sec:anexod:feedback}
% Cada email é um arquivo .tex separado, assim permite o comentário 
Primeira mensagem postada na lista de desenvolvedores do projeto Bacula no qual demonstra os objetivos e solitação de opinião dos desenvolvedores. A idéia inicial era fazer um driver de acesso para o banco de dados IBM DB2.
\begin{VerbatimInput}[frame=single, fontsize=\tiny, numbers=left, label={[10167]\url{http://article.gmane.org/gmane.comp.sysutils.backup.bacula.devel/10167}}]
{../../data/10167.txt}
\end{VerbatimInput}


Linhas 31 a 40 demonstram uma recomendação para a implementação
\begin{VerbatimInput}[frame=single, fontsize=\tiny, numbers=left, label={[10176]\url{http://article.gmane.org/gmane.comp.sysutils.backup.bacula.devel/10176}}]
{../../data/10176.txt}
\end{VerbatimInput}


Linhas 7 a 15 demonstram a preocupação do projeto a respeito da licença das bibliotecas envolvidas. Basicamente, o projeto Bacula possue alguns conflitos com outros softwares que utilizam uma licença proprietária. Esta limitação  impediu a idéia original da contribuição.
\begin{VerbatimInput}[frame=single, fontsize=\tiny, numbers=left, label={[10178]\url{http://article.gmane.org/gmane.comp.sysutils.backup.bacula.devel/10178}}]
{../../data/10178.txt}
\end{VerbatimInput}


\begin{VerbatimInput}[frame=single, fontsize=\tiny, numbers=left, label={[10947]\url{http://article.gmane.org/gmane.comp.sysutils.backup.bacula.devel/10947}}]
{../../data/10947.txt}
\end{VerbatimInput}


Outro desenvolvedor expõe suas opiniões e limitações sobre o desenvolvimento do driver para DB2. Além de sugerir a implementação de uma interface para banco de dados. Neste caso utilizando uma técnica conhecida como Database Abstraction Interface (DBI) no qual encapsula todas as API necessárias para comunicação com o SGBD via uma única API. 
\begin{VerbatimInput}[frame=single, fontsize=\tiny, numbers=left, label={[10948]\url{http://article.gmane.org/gmane.comp.sysutils.backup.bacula.devel/10948}}]
{../../data/10948.txt}
\end{VerbatimInput}


Linhas 66 a 71 demonstram a preocupação do lider do projeto Bacula em relação a licença, reafirmando as licenças permitidas no projeto.
\begin{VerbatimInput}[frame=single, fontsize=\tiny, numbers=left, label={[10949]\url{http://article.gmane.org/gmane.comp.sysutils.backup.bacula.devel/10949}}]
{../../data/10949.txt}
\end{VerbatimInput}


Linhas 24 a 28 definem melhor oque é a DBI, segundo desenvolvedor
\begin{VerbatimInput}[frame=single, fontsize=\tiny, numbers=left, label={[10951]\url{http://article.gmane.org/gmane.comp.sysutils.backup.bacula.devel/10951}}]
{../../data/10951.txt}
\end{VerbatimInput}


Após um período de exploração do conceito de DBI e possível utilização no projeto, retomamos a discussão apresentando a possível solução (linhas 10 a 13), objetivos (linhas 17 a 20) e motivações (linhas 22 a 26).
\begin{VerbatimInput}[frame=single, fontsize=\tiny, numbers=left, label={[10972]\url{http://article.gmane.org/gmane.comp.sysutils.backup.bacula.devel/10972}}]
{../../data/10972.txt}
\end{VerbatimInput}


Aceitação do lider do projeto.
\begin{VerbatimInput}[frame=single, fontsize=\tiny, numbers=left, label={[10975]\url{http://article.gmane.org/gmane.comp.sysutils.backup.bacula.devel/10975}}]
{../../data/10975.txt}
\end{VerbatimInput}


Aprovação da idéia pelo mesmo desenvolvedor que, anteriormente, sugerio a utilização da DBI.
\begin{VerbatimInput}[frame=single, fontsize=\tiny, numbers=left, label={[10980]\url{http://article.gmane.org/gmane.comp.sysutils.backup.bacula.devel/10980}}]
{../../data/10980.txt}
\end{VerbatimInput}


\subsection{Contorno de problemas}\label{sec:anexod:problemas}
A seguir dois exemplos relacionados a um problema encontrado no design do Bacula no qual impossibilitava a utilização da biblioteca libdbi. A situação foi sanada modificando o modo como a função principal da camada de banco de dados do Bacula era chamada. As modificações necessárias foram implementadas pelo gerente do projeto.
\begin{VerbatimInput}[frame=single, fontsize=\tiny, numbers=left, label={[11241]\url{http://article.gmane.org/gmane.comp.sysutils.backup.bacula.devel/11241}}]
{../../data/11241.txt}
\end{VerbatimInput}

\begin{VerbatimInput}[frame=single, fontsize=\tiny, numbers=left, label={[11242]\url{http://article.gmane.org/gmane.comp.sysutils.backup.bacula.devel/11242}}]
{../../data/11242.txt}
\end{VerbatimInput}

\subsection{Submissão do patch libdbi}\label{sec:anexod:submissao}
A seguir, exemplos de submissões feitas ao longo da experiência mostrando os vários tipos de submissões realizadas.
\subsubsection{Exemplo de submissão temporal}\label{sec:anexod:submissaotemporal}
Publicação do patch \patchshort na lista de desenvolvimento. As linhas 23 a 31 mostram as tarefas pendentes.
\begin{VerbatimInput}[frame=single, fontsize=\tiny, numbers=left, label={[11393]\url{http://article.gmane.org/gmane.comp.sysutils.backup.bacula.devel/11393}}]
{../../data/11393.txt}
\end{VerbatimInput}

%\begin{VerbatimInput}[frame=single, fontsize=\tiny, numbers=left, label={[11405]\url{http://article.gmane.org/gmane.comp.sysutils.backup.bacula.devel/11405}}]
{../../data/11405.txt}
\end{VerbatimInput}

\subsubsection{Exemplo de submissão de finalização}\label{sec:anexod:submissaofinal}
% TODO
\subsubsection{Exemplo de submissão de manutenção}\label{sec:anexod:submissaomanutencao}
A seguir mostramos o desenrolar de algumas tarefas pendentes. A primeira tarefa era a implementação do suporte a \textit{batch insert} para melhoramento de performance do Bacula. Inicialmente tinhamos planejado não implementá-la, pelo menos na versão inicial do patch. Entretanto percebemos que era necessário devido ao aumento da performance do software.
\begin{VerbatimInput}[frame=single, fontsize=\tiny, numbers=left, label={[11407]\url{http://article.gmane.org/gmane.comp.sysutils.backup.bacula.devel/11407}}]
{../../data/11407.txt}
\end{VerbatimInput}

Na mesagens seguintes, discutimos sobre alguns problemas na instalação dos scripts de configuração de banco de dados utilizado pelo Bacula. Em resumo, quando o Bacula era compilado com suporte a libdbi, nenhum script de banco de dados era instalado. Felizmente, conseguimos resolver o problema utilizando as sugestões discutidas nos emails.
\begin{VerbatimInput}[frame=single, fontsize=\tiny, numbers=left, label={[11415]\url{http://article.gmane.org/gmane.comp.sysutils.backup.bacula.devel/11415}}]
{../../data/11415.txt}
\end{VerbatimInput}

\begin{VerbatimInput}[frame=single, fontsize=\tiny, numbers=left, label={[11416]\url{http://article.gmane.org/gmane.comp.sysutils.backup.bacula.devel/11416}}]
{../../data/11416.txt}
\end{VerbatimInput}

\begin{VerbatimInput}[frame=single, fontsize=\tiny, numbers=left, label={[11420]\url{http://article.gmane.org/gmane.comp.sysutils.backup.bacula.devel/11420}}]
{../../data/11420.txt}
\end{VerbatimInput}

\begin{VerbatimInput}[frame=single, fontsize=\tiny, numbers=left, label={[11441]\url{http://article.gmane.org/gmane.comp.sysutils.backup.bacula.devel/11441}}]
{../../data/11441.txt}
\end{VerbatimInput}

\begin{VerbatimInput}[frame=single, fontsize=\tiny, numbers=left, label={[11460]\url{http://article.gmane.org/gmane.comp.sysutils.backup.bacula.devel/11460}}]
{../../data/11460.txt}
\end{VerbatimInput}
