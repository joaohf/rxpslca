% Copyright (c) 2008, João Henrique Ferreira de Freitas
% All rights reserved.
% 
% Redistribution and use in source and binary forms, with or without modification,
% are permitted provided that the following conditions are met:
% 
%     * Redistributions of source code must retain the above copyright notice,
%       this list of conditions and the following disclaimer.
%     * Redistributions in binary form must reproduce the above copyright notice,
%       this list of conditions and the following disclaimer in the documentation and/or 
%       other materials provided with the distribution.
%     * Neither the name of the <ORGANIZATION> nor the names of its contributors may
%       be used to endorse or promote products derived from this software without 
%       specific prior written permission.
% 
% THIS SOFTWARE IS PROVIDED BY THE COPYRIGHT HOLDERS AND CONTRIBUTORS "AS IS" AND ANY 
% EXPRESS OR IMPLIED WARRANTIES, INCLUDING, BUT NOT LIMITED TO, THE IMPLIED WARRANTIES
% OF MERCHANTABILITY AND FITNESS FOR A PARTICULAR PURPOSE ARE DISCLAIMED. IN NO EVENT
% SHALL THE COPYRIGHT OWNER OR CONTRIBUTORS BE LIABLE FOR ANY DIRECT, INDIRECT, INCIDENTAL,
% SPECIAL, EXEMPLARY, OR CONSEQUENTIAL DAMAGES (INCLUDING, BUT NOT LIMITED TO, PROCUREMENT
% OF SUBSTITUTE GOODS OR SERVICES; LOSS OF USE, DATA, OR PROFITS; OR BUSINESS INTERRUPTION)
% HOWEVER CAUSED AND ON ANY THEORY OF LIABILITY, WHETHER IN CONTRACT, STRICT LIABILITY,
% OR TORT (INCLUDING NEGLIGENCE OR OTHERWISE) ARISING IN ANY WAY OUT OF THE USE OF THIS
% SOFTWARE, EVEN IF ADVISED OF THE POSSIBILITY OF SUCH DAMAGE.
% 
% $Id$

\section{Anexo B: Taxionomia para entendimento do domínio do problema} \label{sec:anexob}

A seguir um exemplo contendo um macro tema e a respectiva localização do fragmento ou da informação como um todo. É importante observar que juntando todas as informações se torna evidente o processo. Se observarmos cada informação separadamente, verificamos pouca relação com a engenharia de software tradicional.

\begin{itemize}
\item Portal oficial do projeto: \small\url{http://www.bacula.org}
\item Documentações oficiais e presentes no portal, ou seja, qual documento que clareia algum ponto relacionado diretamente ou indiretamente ao software. Ex.:
 \subitem documentação de usuário: \\                                    \small\url{/manuals/en/console/console/index.html}
 \subitem documentação de configuração: \\ \small\url{/manuals/en/install/install/index.html}
 \subitem documentação de desenvolvedores:\\ \small\url{/manuals/en/developers/developers/index.html}
\item Relatórios de status, atribuições, projetos e trabalhos: \\
\subitem projetos listados: \\
 \small\url{/en/?page=projects}
\subitem recursos suportados pelo software:\\
 \small\url{/en/dev-manual/Current_State_Bacula.html}
\subitem apresentações em palestras: \\
\small\url{/en/?page=presentations}
\subitem notas de release: \\
 \small\url{/en/?page=presskits}
\subitem processo para novas features: \\
 \small\url{/en/?page=feature-request}
\subitem novidades do projeto: \\
\small\url{/en/?page=news}

\item Comunicação assincrona entre participantes:
\subitem postadas em listas de discussão: \\
\small\url{/en/?page=maillists}
\subitem busca em arquivos das listas de discussão: \\ \small\url{http://news.gmane.org/search.php?match=bacula} e \small\url{http://marc.info/}
\item Repositório e código fonte: \\
\small\url{http://sourceforge.net/svn/?group_id=50727}
\item Processos gerais:
\subitem ciclo de desenvolvimento: \\ \small\url{/manuals/en/developers/developers/Development_Cycle.html}
\subitem submissão de patchs: \\ \small\url{/manuals/en/developers/developers/Bacula_Code_Submiss_Project.html}
\item Ferramentas de desenvolvimento do projeto: \\
 \subitem controle de versão: \\ \small\url{/manuals/en/developers/developers/SVN_Usage.html}
 \subitem recursos de desenvolvimento: \\ \small\url{/manuals/en/developers/developers/Developing_Bacula.html}
Gerencia de configuração: \\ \small\url{/manuals/en/developers/developers/Steps_Take_Porting.html}
\subitem bug reporting: \\
 \small\url{/en/?page=bugs}
\subitem framework para testes: \\
 \small\url{/manuals/en/developers/developers/Bacula_Regression_Testing.html} e \small\url{http://bacula.svn.sourceforge.net/viewvc/bacula/trunk/regress}
\subitem scripts de teste e resultados: \\ \small\url{http://regress.bacula.org:8081/Bacula/Dashboard/}                                    \end{itemize}



